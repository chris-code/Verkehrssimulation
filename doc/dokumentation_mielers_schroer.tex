\documentclass[11pt, a4paper]{article}

\usepackage[utf8]{inputenc}
\usepackage{graphicx}
\graphicspath{ {images/} }
\usepackage{mathtools}
\usepackage{amssymb}
\usepackage{amsmath}
\usepackage[english, ngerman]{babel}
\usepackage{cite}
\usepackage{bibgerm}
\usepackage{fullpage}
\usepackage[top=1.5cm,bottom=1.5cm,left=3.5cm,right=2.5cm,headsep=1.5cm,includeheadfoot]{geometry}
\usepackage{tabularx}
\usepackage{eurosym}
\usepackage{enumitem}
\usepackage{multicol}
\usepackage{tikz}
\usepackage{tkz-euclide}
\usepackage{pgfplots}
\usepackage{pdflscape}
\usepackage{acronym}
\usepackage{blindtext}
\usepackage{ifthen}
\usepackage{setspace}
\usepackage{cancel}
\usepackage{color}
\usepackage{listings}
\usepackage{comment}
\usepackage{xcolor}
\usepackage{colortbl}

\usepackage{fancyhdr}
\pagestyle{fancy}

\fancyhf{} % clear all
\fancyhead[L]{\leftmark}
\fancyfoot[C]{-- \thepage{} --}
%\setlength{\headheight}{15pt}
\renewcommand{\headrulewidth}{0.5pt}
\renewcommand{\footrulewidth}{0pt}
\setlength{\skip\footins}{0.7cm}

\usetikzlibrary{graphs}
\usetikzlibrary{positioning}

\onehalfspacing
\setlength\parindent{0pt}

\everymath{\displaystyle}

\allowdisplaybreaks

\definecolor{AI-BLUE}{rgb}{0,0.57,0.87}

% Eigene Befehle
\newcommand\q[1]{\glqq{}#1\grqq{}}
\renewcommand\equiv{\Leftrightarrow}
\newcommand\vertequal[2]{\underset{\underset{#2}{\parallel}}{#1}}
\newcommand\cif{\text{if }}
\newcommand\abs[1]{\left|#1\right|}
\newcommand\norm[1]{\abs{\abs{#1}}}
\newcommand\diff[1]{\text{ d#1}}
\newcommand\av[1]{\left\langle#1\right\rangle}
\newcommand\ev[1]{\mathbb{E}\left(#1\right)}
\newcommand\br[1]{\left(#1\right)}
\newcommand\ubr[2]{\underbrace{#1}_{#2}}
\newcommand\quer[1]{\overline{#1}}
\newcommand\setequal{\overset{!}{=}}
\newcommand\dint{\displaystyle \int}
\newcommand\dsum{\displaystyle \sum}
\newcommand\dprod{\displaystyle \prod}
\newcommand\closedInt[2]{\left[#1,#2\right]}
\newcommand{\checkbox}{\Large \Square \normalsize \hspace{0.4cm}}

\newcommand\myref[1]{\ref{#1} (S. \pageref{#1})}
\newcommand\myrefcomma[1]{\ref{#1}, S. \pageref{#1}}


\begin{document}

\thispagestyle{empty}

\pagenumbering{Roman}

\setlength{\hoffset}{-0.5cm} % center title page

\begin{titlepage}
    \begin{center}
    \vphantom{0cm}
    \LARGE \textbf{Dokumentation}\\
    \vspace{3cm}
    \normalsize
    Dokumentation für Simulationstechnik \\
    im Master-Studiengang \textcolor{AI-BLUE}{[Angewandte Informatik]}\\
    an der Ruhr-Universität Bochum\\
    im Wintersemester 2014/15\\
    \vspace{4cm}
    \huge \textbf{Verkehrssimulation in C++} \\
    \vspace{4cm}
    \normalsize
    \textbf{Projektteilnehmer:}\\
    B. Sc. Christian Andreas Mielers (108011204956)\\
    B. Sc. Phil Yannick Schrör (108011214024)\\
    \vspace{2cm}
    \textbf{Projektbetreuer:}\\
    M. Sc. Markus Scheffer
    \end{center}
\end{titlepage}

\newpage

\tableofcontents

\newpage

\pagenumbering{arabic}

\section{Einleitung}
\label{sec:einleitung}

\newpage

\section{Annahmen}
\label{sec:ansatz}

\newpage

\section{Umsetzung}
\label{sec:umsetzung}
\subsection{Einspurige Straße}
\subsection{Mehrspurige Straße}
\subsection{Kreisverkehr (und beliebige Straßenverläufe)}

Ein dritter Teil der Aufgabenstellung bestand darin, den Verkehrsfluss eines Kreisverkehrs zu simulieren. Dies ist ein ungleich schwierigeres Problem, da es hier an einigen Stellen Verzweigungen und Zusammenführungen geben kann, bei denen Kollisionsfreiheit gewährleistet werden muss. Zwar kann man den Spurwechsel auf der mehrspurigen Straße auch als Verzweigung/Zusammenführung betrachten, allerdings weist dieser Gleichmäßigkeit über alle Zellen hinweg auf, d.h. die Regeln können einmalig, relativ zur Zellposition und den Geschwindigkeiten der Fahrzeuge festgelegt werden. Im Kreisverkehr hingegen verhalten sich einige Zellen deutlich anders als andere, beispielsweise hinsichtlich zulässiger Geschwindigkeit oder Anzahl der Richtungen, in die man sich bewegen kann. In diesem Zusammenhang müssen auch Vorfahrtsregeln beachtet werden. Darüber hinaus haben Ereignisse in einer Zelle (z.B. die Entscheidung, eine Ausfahrt zu nehmen) Auswirkungen auf die Handlungsmöglichkeiten der Autos in den umliegenden Zellen. Diese Anforderungen machen einen erheblich dynamischeres Vorgehen bei der Berechnung der Bewegungen der Fahrzeuge erforderlich. Eine Lösung muss also folgende Anforderungen erfüllen:

\begin{enumerate}
	\item Kollisionsfreiheit
	\item Vorfahrtsregeln
	\item Geschwindigkeitsbegrenzungen
\end{enumerate}

Grundsätzlich wäre eine vergleichsweise simple Lösung möglich, die die Anforderungen dieses spezifischen Szenarios erfüllt. Hierbei würde man die kritischen Stellen, wie die Zu- und Abfahrtszellen zum/vom Kreisverkehr, gesondert behandeln und dafür eine andere Logik implementieren. Es ist offensichtlich, dass eine solche Lösung hochgradig abhängig vom der Problemstellung ist und man daher für jedes Szenario manuelle Anpassungen vornehmen muss. Dies ist nicht zufriedenstellend, weswegen wir uns ein Verfahren zum Ziel gesetzt haben, dass von Spezifika des Scenarios (wie der konkreten Position der Abzweigungen) unabhängig ist.

Strukturell behalten wir das Layout eines Grids bei, in dem die Zellen der Simulation mit x- und y-Koordinaten gespeichert werden. Dies dient allerdings nur der Vereinfachung der Visualisierung. Konzeptionell besteht unsere Straßenkarte aus losen Zellen, die Verweise auf ihre Vorgänger- und Nachfolgerzellen beinhalten. Somit entspricht das Layout unserer Lösung einem gerichteten, nicht notwendigerweise kreisfreien Graphen.

Zunächst betrachten wir die Anforderung der Kollisionsfreiheit. Es soll, analog zum Nagel-Schreckenberg Modell \cite{nagel-schreckenberg} erreicht werden, dass in keinem Simulationsschritt eine Zelle von mehr als einem Auto befahren wird. Ein einfacher Weg dies zu erreichen besteht darin, alle Fahrzeuge sequenziell durchzugehen und für jedes die Zellen, die es abfahren möchte zu markieren. Hier geht man nur soweit, wie das Auto mit seiner aktuellen Geschwindigkeit fahren kann. Dabei wird das Markieren gestoppt, sobald eine Zelle erreicht wird, die bereits von einem anderen Auto markiert wurde. Nachdem alle Fahrzeuge ihren Weg markiert haben, wird für jedes die Anzahl der Markierungen gezählt und seine Geschwindigkeit auf diese Zahl gedeckelt.

Bei diesem Ansatz ergibt sich allerdings das Problem, dass Vorfahrtsregeln missachtet werden. Wird zuerst ein Fahrzeug bewegt, dass dann von einer Nebenstraße aus in eine Hauptstraße eingibt und dort Zellen markiert, blockiert es dort möglicherweise ein Vorfahrt habendes Fahrzeug. Es gibt keine Möglichkeit präemtiv zu bestimmen, welches Fahrzeug Vorfahrt haben wird. Das kann man unter Anderem daran sehen, dass die Vorfahrts-Relation nicht transitiv ist, weswegen keine Ordungsrelation auf den Fahrzeugen etabliert werden kann. Daher muss es eine Möglichkeit geben, Markierungen zu überschreiben. Der Ansatz wird also so erweitert, dass ein Fahrzeug nur dann mit dem Markieren aufhört, wenn es an einer Zusammenführung auf eine Markierung trifft, ohne an der Zusammenführung Vorfahrt gehabt zu haben. Hat es Vorfahrt, überschreibt es die Markierungen einfach. Die Vorfahrt wird über die Reihenfolge der Vorgängerverweise einer Zelle bestimmt, ist also eine lokale Eigenschaft. Externe Informationen sind nicht erforderlich. Da die Geschwindigkeit der Fahrzeuge erst berechnet wird nachdem alle Autos abgearbeitet wurden, lassen sich mit dieser Methode Vorfahrtsregeln berücksichtigen. Die Kollisionsfreiheit bleibt dabei erhalten.

Ein Problem beim Überschreiben ist jedoch, dass 'Stummel' vorheriger Markierungen übrig bleiben können. Daher müssen, wenn eine Markierung eines Fahrzeugs entfernt wird, alle Markierungen des Fahrzeugs entfernt werden. Später sind dann die Markierungen des Fahrzeuges neu zu bestimmen.

Zuletzt muss noch die Anforderung des Einhaltens von Geschwindigkeitsbegrenzungen erfüllt werden. Dazu wird jede Zelle mit einer Maximalgeschwindigkeit versehen. Betritt ein Fahrzeug eine Zelle, wird die maximale Anzahl an Zellen, die es darüber hinaus noch markieren kann auf die Maximalgeschwindigkeit der Zelle gedeckelt. Somit ist gewährleistet, dass kein Fahrzeug an keiner Zelle die zulässige Geschwindigkeit überschreitet.

Diese Überlegungen führen zu folgendem Algorithmus:
\lstinputlisting[numbers=right]{src/streetMapAlgorithm}
wobei r die verbleibenden Schritte des Autos runterzählt. Damit haben wir einen Algorithmus erreicht, der beliebige Verkehrsnetze Kollisionsfrei unter Berücksichtigung von Vorfahrtsregeln und Geschwindigkeitsbegrenzungen simulieren kann. Im Kontrast zur mehrspurigen Simulation oben kann dieses Verfahren zwar ebenfalls mit mehreren Spuren umgehen, das Rechtsfahrgebot und das Verbot, rechts zu überholen müssten aber separat implementiert werden.

Eine schöne Eigenschaft dieser Herangehensweise ist, dass wir Quellen und Senken von Fahrzeugen umsonst bekommen. Als Quelle wird schlicht jede Zelle definiert, die zwar Nachfolger, aber keine Vorgänger hat. Analog ist jede Zelle, die Vorgänger, aber keine Nachfolger hat eine Senke. Jeder Zelle kann darüber hinaus eine Autoerzeugungswahrscheinlichkeit zugeordnet werden, mit der in der Zelle pro Simulationsschritt ein neues Auto entsteht sofern sie leer ist. Ein Auto, was eine Senke überfährt verschwindet hingegen.

Nun bleibt noch die Frage zu klären, wie an einer Verzweigung die Entscheidung zu treffen ist, in welche Richtung ein Fahrzeug fahren sollte (vgl. Zeile 7). Konzeptionell wäre es möglich, jedem neu hinzugefügten Fahrzeug ein Ziel zuzuweisen und einen (randomisierten) Pfadfindungsalgorithmus zu nutzen, um die Entscheidungen an Verzweigungen zu treffen. Da Fahrzeuge in der Regel ein festes Ziel haben, wäre dies der realistischste Ansatz. Zur Vereinfachung der Implementierung entscheiden wir uns allerdings dafür, die Entscheidung an jeder Verzweigung zufällig zu treffen, wobei für jeden Nachfolger individuelle Wahrscheinlichkeiten festgelegt werden können. Da wir nur sehr einfache Straßennetze mit wenigen Verzweigungen (die insbesondere nicht verschachtelt sind) betrachten, erachten wir das als eine ausreichend gute Annäherung. Für komplexere Szenarien werden allerdings komplexere Entscheidungssysteme notwendig.

%TODO: Nicht-transitivität der Vorfahrtsrelation illustrieren
%TODO: Problem: wenn wir markierungen überschreiben, können 'Stummel' überbleiben, die andere Fahrzeuge blockieren.

\newpage

\section{Ergebnisse}
\label{sec:ergebnisse}

\newpage

\section{Fazit}
\label{sec:fazit}

\blindtext asdf \cite{mehrspurig}\\
In Kapitel \myref{sec:einleitung} steht Shit. In Foobar (\myrefcomma{sec:ansatz}) bla

\newpage

\addcontentsline{toc}{section}{Literatur}
\bibliography{ref}{}
\bibliographystyle{alpha}

\end{document}